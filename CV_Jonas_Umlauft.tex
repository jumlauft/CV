%% start of file `template.tex'.
%% Copyright 2006-2015 Xavier Danaux (xdanaux@gmail.com).
%
% This work may be distributed and/or modified under the
% conditions of the LaTeX Project Public License version 1.3c,
% available at http://www.latex-project.org/lppl/.

\newif\ifde
\newif\ifen

% Put \entrue for English or \detrue for German version
\detrue   
% Modify moderncvstylebanking.sty to update header
% --> \moderncvhead{3} for English
% --> \moderncvhead{1} for German


\documentclass[11pt,a4paper,sans]{moderncv}        % possible options include 
%font size ('10pt', '11pt' and '12pt'), paper size ('a4paper', 'letterpaper', 
%'a5paper', 'legalpaper', 'executivepaper' and 'landscape') and font family 
%('sans' and 'roman')
\usepackage{mathabx}
\usepackage[gen]{eurosym}
\usepackage[sorting = none, style = numeric,%dashed=false,
giveninits=true,maxbibnames=99,doi=false,isbn=false,url=false]{biblatex}
\addbibresource{mybib.bib}
\AtEveryBibitem{\clearfield{month}}
\AtEveryBibitem{\clearfield{day}}
% moderncv themes
\moderncvstyle{banking}                             % style options are 
%'casual' 
%(default), 'classic', 'banking', 'oldstyle' and 'fancy'
\moderncvcolor{blue}                               % color options 'black', 
%'blue' (default), 'burgundy', 'green', 'grey', 'orange', 'purple' and 'red'
%\renewcommand{\familydefault}{\sfdefault}         % to set the default font; 
%%use '\sfdefault' for the default sans serif font, '\rmdefault' for the 

\newcommand{\DE}[1]{%
	\ifde\selectlanguage{ngerman}#1\fi}
\newcommand{\EN}[1]{%
	\ifen\selectlanguage{english}#1\fi}

\usepackage[english,ngerman]{babel}

% adjust the page margins
\usepackage[scale=0.75]{geometry}
%\setlength{\hintscolumnwidth}{3cm}                % if you want to change the 
%%width of the column with the dates
%\setlength{\makecvheadnamewidth}{10cm}            % for the 'classic' style, 
%%if you want to force the width allocated to your name and avoid line breaks. 
%%be 
%%careful though, the length is normally calculated to avoid any overlap with 
%%your personal info; use this at your own typographical risks...

% personal data
\name{Jonas}{Umlauft}
%\title{Resumé title}                               % optional, remove / 
%%comment 
%the line if not wanted
\DE{\dateofbirth{$^\Asterisk$10.04.1990}}
\address{Mohnweg 13}{86391 
\DE{Stadtbergen}\EN{Stadtbergen}}{\EN{Germany}}% 
%optional, 
%remove / comment 
%the line if not wanted; the "postcode city" and "country" arguments can be 
%omitted or provided empty
\phone[mobile]{+49~(178)~134~7603$\ $}                   % optional, 
\github{www.github.com/jumlauft}                               % 
\email{jumlauft@gmail.com}                               % optional, remove / 
%comment 
%the line if not wanted
\homepage{www.ei.tum.de/itr/umlauft}                   % 

\photo[64pt][0.4pt]{jonas.jpg}                       % optional, remove / 

\renewcommand*{\bibliographyitemlabel}{[\arabic{enumiv}]}
\newcommand{\CC}{C\nolinebreak\hspace{-.05em}\raisebox{.4ex}{\tiny 
\textbf{+}}\nolinebreak\hspace{-.10em}\raisebox{.4ex}{\tiny \textbf{+}}}

%   to redefine the bibliography heading string ("Publications")
%\renewcommand{\refname}{Articles}
\def\makenamesetup{%
	\def\bibnamedelima{~}%
	\def\bibnamedelimb{ }%
	\def\bibnamedelimc{ }%
	\def\bibnamedelimd{ }%
	\def\bibnamedelimi{ }%
	\def\bibinitperiod{.}%
	\def\bibinitdelim{~}%
	\def\bibinithyphendelim{.-}}    
\newcommand*{\makename}[2]{\begingroup\makenamesetup\xdef#1{#2}\endgroup}

\newcommand*{\boldname}[3]{%
	\def\lastname{#1}%
	\def\firstname{#2}%
	\def\firstinit{#3}}
\boldname{}{}{}

% Patch new definitions
\renewcommand{\mkbibnamegiven}[1]{%
	\ifboolexpr{ ( test 
	{\ifdefequal{\firstname}{\namepartgiven}} or test 
	{\ifdefequal{\firstinit}{\namepartgiven}} ) and test 
	{\ifdefequal{\lastname}{\namepartfamily}} }
	{\mkbibbold{#1}}{#1}%
}

\renewcommand{\mkbibnamefamily}[1]{%
	\ifboolexpr{ ( test 
	{\ifdefequal{\firstname}{\namepartgiven}} or test 
	{\ifdefequal{\firstinit}{\namepartgiven}} ) and test 
	{\ifdefequal{\lastname}{\namepartfamily}} }
	{\mkbibbold{#1}}{#1}%
}

\boldname{Umlauft}{Jonas}{J.}
% bibliography with mutiple entries
%\usepackage{multibib}
%\newcites{book,misc}{{Books},{Others}}
%----------------------------------------------------------------------------------
%            content
%----------------------------------------------------------------------------------





\newcommand{\inQM}[1]{\ifde \glqq #1\grqq\fi\ifen ``#1''\fi}

\newcommand{\TUM}{\textnormal{ \DE{Technische Universität 
München}\EN{Technical 
University of Munich}}}
\begin{document}
%\begin{CJK*}{UTF8}{gbsn}                          % to typeset your resume in 
%%Chinese using CJK
%-----       resume       
%---------------------------------------------------------
\ifen 
\selectlanguage{english}
\fi
\
\EN{\def\sepspace{0.15cm}}
\DE{\def\sepspace{0.1cm}}
\makecvtitle

\EN{\vspace{-1cm}}
\DE{\vspace{-1.5cm}}
\section{\DE{Ausbildung}\EN{Education}}
\cventry{}{\vspace{-0.5cm}}
		{\DE{Dr.-Ing. Elektro-/Informationstechnik,}\EN{Dr.-Ing. Electrical 
		Engineering,}\TUM}
		{05/2015 -- 07/2020}{}
		{\begin{itemize}
%				\item \DE{Lehrstuhl für informationstechnische Regelung, 
%				Betreuerin: Prof. Sandra Hirche}
%						\EN{Chair for Information-oriented control, Supervisor: 
%						Prof. Sandra Hirche}
				\item \DE{Kombination von Regelungstechnik und maschinellem 
					Lernen um die Sicherheit von selbstlernenden, 
					autonomen Systemen zu garantieren}
					\EN{Combined control engineering and machine learning to 
					ensure safety of self-learning autonomous systems}
				\item \DE{Titel der Doktorarbeit}\EN{Title of dissertation}: 
					\inQM{Safe Learning Control for Gaussian Process Models},
					\EN{Grade:}\DE{Note:} \textit{\inQM{summa cum laude}}
		\end{itemize}}  % 
\vspace{\sepspace}
\cventry{}{\vspace{-0.5cm}}
	{\DE{M. Sc. Elektro-/Informationstechnik,}\EN{M. Sc. Electrical 
	Engineering,}\TUM}
 	{10/2013 -- 03/2015}{}
 	{\begin{itemize}
 		\item \DE{Spezialisierung in Regelungstechnik, Robotik und Optimierung}
 			\EN{Specialized in control theory, robotics and optimization}
 		\item \DE{Note: 1,1 \inQM{mit Auszeichnung} (1,0 am besten, 5,0 
 			am schlechtesten, top 6\%)}
 			\EN{Grade: 1.1 \inQM{with high distinction} (1.0 is best, 5.0 is 
 			worst, best 6 \%)}
	 \end{itemize}}  % 
\vspace{\sepspace}
\cventry{}{\vspace{-0.5cm}}
		{\DE{Auslandssemester}\EN{Semester Abroad},
		\textnormal{National University of Singapore, Singapore}}
		{08/2014 -- 12/2014}{}
		{\begin{itemize}
				\item \DE{Vertiefungskurse in maschinellem Lernen und 
				Informationstheorie}
			\EN{Specialized in machine learning and information 
					theory}
		\end{itemize}} 
\vspace{\sepspace}
\cventry{}
		{\vspace{-0.5cm}}
		{\DE{Auslandssemester}\EN{Semester Abroad}, 
			\textnormal{University of Cambridge, UK}}
		{02/2014 -- 08/2014}{}
		{\begin{itemize}
				\item \DE{Masterarbeit}\EN{Master's Thesis}: 
					\inQM{Probabilistic Models for Nonlinear System 
					Identification and Control}, 
				\DE{Note: 1,0}\EN{Grade: 1.0}
				\item \DE{Computational and Biological Learning Lab, 
					Betreuer: Carl Rasmussen}\EN{Worked on model-based 
					reinforcement learning algorithms supervised by Carl 
					Rasmussen} 
		\end{itemize}} 
\vspace{\sepspace}
\cventry{}{\vspace{-0.5cm}}
	{\DE{B. Sc. Elektro-/Informationstechnik}\EN{B. Sc. Electrical 
	Engineering},\TUM}
	{04/2011 -- 09/2013}{}
	{\begin{itemize}
			\item \DE{Bachelorarbeit}\EN{Bachelor's Thesis}: 
			\inQM{Dynamic 
			Movement 
				Primitives for Cooperative Robotic Manipulation}$\!$, 
				\DE{Note: 1,0}\EN{Grade: 1.0}
			\item \DE{Note: 1,4 \inQM{sehr gut} (1,0 am besten, 5,0 
				am schlechtesten, top 5\%)}
			\EN{Grade: 1.4 \inQM{with distinction} (1.0 is best, 5.0 is 
				worst, best 5 \%)}
	\end{itemize}} 

\vspace{\sepspace}
\cventry{}
	{\vspace{-0.5cm}}
	{\DE{Studium Elektrotechnik}\EN{Studies Electrical 
	Engineering}, \textnormal{University of Hawaii, USA}}
	{08/2009 -- 05/2011}{}
	{\begin{itemize}
%			\item \DE{Stipendium als Mitglied des Volleyballteams der 
%				Universität}\EN{Started  a B.Sc degree in electrical 
%				engineering}
			\item \DE{Stipendium als Mitglied des Volleyballteams der 
			Universität}\EN{Received scholarship as student athlete volleyball}
	\end{itemize}} 

%\cventry{}{\vspace{-0.5cm}}
%		{\DE{Abitur}\EN{High School Diploma}, \textnormal{Landschulheim 
%		Kempfenhausen, Berg }}
%		{06/2009}{}
%		{} 

	 
%arguments 3 to 6 can be left empty
%\cventry{year--year}{Degree}{Institution}{City}{\textit{Grade}}{Description}

%\section{Master thesis}
%\cvitem{title}{\emph{Title}}
%\cvitem{supervisors}{Supervisors}
%\cvitem{description}{Short thesis abstract}

\section{\EN{Working Experience}\DE{Berufserfahrung}}
%\subsection{Vocational}
\cventry{}{\vspace{-0.5cm}}
		{\DE{Wissenschaftlicher Mitarbeiter}\EN{Research Associate},\TUM}
		{05/2015\hspace{2.5pt} -- \hspace{2.5pt}\DE{heute}\EN{present}}{}
		{\begin{itemize}%
			\item \DE{Forschungsprojekt \inQM{Control based on Human Models} am
						Lehrstuhl für informationstechnische Regelung}
					\EN{Contributed to research project $\!$\inQM{Control based 
						on Human Models}$\!$ at Chair for Information-oriented 
						Control}
			\item \DE{Lehrverantwortung für Vorlesungen, Praktika und 
					Prüfungen, Betreuung von über 20 
					Studentenprojekten}\EN{Took responsibility for lectures, 
					lab 
					courses, tutorials, exams and supervised over 20 
					student projects}
			\item \DE{Erfolgreicher Antrag für internationales 
			Forschungsprojekt mit ca.$\!$ 7$\!$ Mio$\!$ \euro{} 
			Drittmittelförderung 
			von der~EU}\EN{Coauthored grant 
					proposal for an international project funded by the EU with 
					over 7 million \euro{}}
		\end{itemize}}
\vspace{\sepspace}	
\cventry{}{\vspace{-0.5cm}}
	{\DE{Werkstudent}\EN{Working Student}, \textnormal{BMW 
	Group, \DE{München}\EN{Munich}}}
	{10/2012 -- 03/2013}{}
	{\begin{itemize}%
%		\item \DE{Abteilung Absicherung von Hochvoltspeichern}
%			  \EN{Department Test and Validation of High Voltage Batteries}
		\item \DE{Entwurf von Testspezifikationen, Vorbereitung und Auswertung 
		der Tests für Hochvoltspeicher}
			  \EN{Designed test specification, prepared experimental 
			  		setups and analyzed test results for high voltage batteries}
	\end{itemize}}
%\cventry{}{\vspace{-0.5cm}}
%		{\DE{Studentische Hilfskraft}\EN{Student Assistant},\TUM}
%		{10/2012 -- 01/2013}{}
%		{\begin{itemize}%
%			\item \DE{Forschungsprojekt zur Mensch-Roboter-Interaktion am 
%			Lehrstuhl für Steuerungs- und Regelungstechnik}
%				\EN{Research project on Human-Robot Interaction at the 
%				Institute of Automatic Control Engineering}
%			\item \DE{Entwurf eines Reglers für kooperatives Schwingen und 
%					Untersuchungen zu chaotischem Verhalten}
%				\EN{Investigation of cooperative swing-up of a pendulum and the 
%				occurrence of chaotic behavior}
%		\end{itemize}}
\vspace{\sepspace}
\cventry{}{\vspace{-0.5cm}}
{\DE{Werkstudent}\EN{Working Student}, \textnormal{Siemens 
AG, \DE{München}\EN{Munich}}}
{10/2011 -- 09/2012}{}
{\begin{itemize}%
%		\item \DE{Abteilung Corporate Technology, Software Architecture and 
%		Platforms}
%		\EN{Department Corporate Technology, System Architectures and 
%		Platforms}
		\item \DE{Recherche und Implementierung eines Prototyps in HTML5, 
		Abteilung System Architectures and Platforms}
			\EN{Implemented prototype and researched on HTML5 for the 
				department System Architectures and Platforms }
\end{itemize}}


\section{\DE{Engagement}\EN{Commitments}}
\cventry{}{\vspace{-0.5cm}}
	{\DE{Sprecher des Graduate Councils}\EN{Graduate 
	Council Speaker},\TUM}
	{10/2016 -- 09/2017}{}
	{\begin{itemize}%
		\item \DE{Vorsitz des Councils bestehend aus ca. 50 
			VertreterInnen der Promovierenden mit Budget von über 
			20.000\euro}\EN{Chaired the council consisting 
			of approx.~50 doctoral representatives with a 
			budget of over 20,000\euro}
		\item \DE{Vertretung der hochschulpolitischen Interessen von ca. 5000	
				Promovierenden der TUM}\EN{Represented the university political 
				interests of over 5000 doctoral candidates}
		\item \DE{Mitglied des Senats, des Hochschulrats und des Vorstands 
				der TUM Graduate School}\EN{Joined the Board of the TUM 
			Graduate School, the TUM Senate, and the TUM Board of Trustees}
		%			\item \DE{Einführung des \inQM{TUM Supervisory Award} 
		%(Auszeichnung 
		%			für gute Betreuung von 	Promovierenden)}\EN{Foundation of 
		%the 
		%			\inQM{TUM Supervisory Award} awarding outstanding 
		%			supervision of PhD candidates}
	\end{itemize}}

\newpage
%\DE{\vspace{\sepspace}

\cventry{}{\vspace{-0.5cm}}
{\DE{Doktorandenvertreter}\EN{Doctoral 
Representative},\TUM}
	{08/2015 -- 07/2020 }{}
	{\begin{itemize}%
%		\item \DE{Vertretung der hochschulpolitischen Interessen der
%			Promovierenden der Elektro-/Informationstechnik}
%		\EN{Representation of the university political interests of 
%			doctoral candidates in electrical engineering}	
%		\item \DE{Mitglied im Graduate 
%			Council und im Vorstand des Fakultätsgraduiertenzentrums}
%		\EN{Member of the Graduate Council and the board of the Faculty 
%			Graduate Center}
		\item \DE{Mitglied des Graduate Councils und Leiter der Arbeitsgruppe 
				\textit{Betreuung} (3-4 personen) }
			\EN{Joined the Graduate Council and led the working group 
				\textit{supervision} (3-4 people)}
		\item \DE{Gründer des \inQM{TUM Supervisory Award} (Auszeichnung für 	
				gute Betreuung von Promovierenden)}\EN{Founded the 
				\inQM{TUM Supervisory Award} which awards 5,000\euro{} for
				outstanding supervision of PhD students }
		\item \DE{Organisator einer Netzwerkveranstaltung für alle 		
					Promovierenden der TUM (ca. 800 Teilnehmer)}\EN{Organized a 
					network event for all doctoral candidates of TUM (approx. 
					800 participants)}	
	\end{itemize}}

%\cventry{}{\vspace{-0.5cm}}
%		{\DE{Vertretung wissenschaftlicher Mitarbeiter}\EN{Representative of 
%		scientific staff},\TUM}
%		{10/2019 -- heute}{}
%		{\begin{itemize}%
%				\item \DE{Mitglied im Fakultätsrat Elektro-/Informationstechnik}
%				\EN{Member of the departement council electrical engineering}
%		\end{itemize}}
%\vspace{\sepspace}
%\cventry{}{\vspace{-0.5cm}}
%		{\DE{Teilnehmer Model United Nations 
%		(MUN)}\EN{Participant 
%		Model United Nations (MUN)}}
%		{10/2015 -- 04/2016}{}
%		{\begin{itemize}%
%			\item \DE{Vertreter Qatars im Menschrechtsrat bei der IsarMUN 2015
%					 in München}\EN{Represented Qatar in the human rights 
%					council at IsarMUN 2015 in Munich}
%			\item \DE{Vertreter Vietnams im Rechtsausschuss bei der WorldMUN 	
%					2016 in Rom}\EN{Represented Vietnam in the legal committee 
%					at WorldMUN 2016 in Rome}
%		\end{itemize}}

\vspace{\sepspace}
\cventry{}{\vspace{-0.5cm}}
		{\DE{Stipendiat bei Manage\&More}\EN{Participant at 
		Manage\&More},
		\textnormal{UnternehmerTUM GmbH, 
		\DE{München}\EN{Munich}}}
		{10/2012 -- 03/2014}{}
		{\begin{itemize}%
				\item \DE{Unternehmerisches Fortbildungsprogramm zur 
						Förderung von Führungs- und 
						Projektmanagementfähigkeiten}\EN{Improved my project	
						management skills and gained a
						product-driven mindset 
						}
				\item \DE{Leiter eines Team von 5 Studenten in einem 
				Innovationsprojekte mit der BMW Group}\EN{Led a team of 5 
				people in an innovation 
				project with BMW}
		\end{itemize}}


%\cventry{}{\vspace{-0.5cm}}
%		{\DE{Teilnehmer Projekt eCARus}\EN{Participant eCARus Project}}
%		{10/2011 -- 03/2013}{}
%		{\begin{itemize}
%				\item \DE{Studierende der TUM entwickeln in 
%				Eigenverantwortung 
%		Elektrofahrzeuge}\EN{Students design and build 
%		independently an 
%		electrical vehicle}
%				\item \DE{Mitglied in den Teams Bordnetz und 
%				Karosserie, IT und mobile Energieversorgung}
%					\EN{Member of the teams vehicle power 
%					and body, IT and mobile energy supply}
%	\end{itemize}}

\vspace{\sepspace}
\cventry{}{\vspace{-0.5cm}}
	{\DE{Leistungssportler Volleyball}\EN{Athlete Volleyball}}
	{12/2004 -- 04/2011}{}
	{\begin{itemize}
			\item \DE{Nationalmannschaft, Teamkapitän in der Bundesliga, 	
					All-American team der NCAA (USA)}\EN{Played for the German 
					national team and was selected to the NCAA All-American 
					team (USA)}
			\item \DE{Lizenz als Übungsleiter und Vereinsmanager 
					C}\EN{Qualified as volleyball trainer and club manager 
				C-level}
	\end{itemize}}
	
\section{\EN{Awards}\DE{Auszeichnungen}}
	\cventry{}{\vspace{-0.5cm}}
{Kurt-Fischer \DE{Promotionspreis}\EN{PhD Award}}
{2020}{}
{\DE{Verliehen durch die Fakultät für Elektrotechnik und Informationstechnik 
für eine herausragende Doktorarbeit}
	\EN{Awarded by the Department of Electrical and Computer 
		Engineering for an exceptional thesis}}
	\cventry{}{\vspace{-0.5cm}}
		{IEEE Conference on Decision and Control Outstanding Student Paper 
			Award}
		{2018}{}
		{\DE{Auswahl von drei Beiträgen (aus über 2000) erfolgt durch 
		Expertenkommission}
		\EN{Selected from over 2000 submissions (with two others) by an expert 
		committee}}
\vspace{\sepspace}	
\cventry{}{\vspace{-0.5cm}}
		{\DE{Stipendiat des Max Weber-Programms (Studienstiftung)}\EN{Scholar 
		of the Max Weber-Programm}}
		{10/2013 -- 03/2015}{}
		{\DE{Auswahl aufgrund hervorragender Studienleistungen 	
			(top 3\% des Jahrgangs) und eines persönlichen 
			Gesprächs}\EN{Granted based on excellent 
			grades (best 3\% of cohort) and a personal 
		interview}}	

%\cventry{}
%		{\vspace{-0.5cm}}
%		{\DE{Stipendium der Heinrich und Lotte Mühlfenzel 
%		Stiftung}\EN{Scholarship of the Heinrich und Lotte 
%Mühlfenzel 
%		Foundation}}
%		{2014}{}{}	
%\cventry{}
%		{\vspace{-0.5cm}}
%		{\DE{Stipendium für Ingenieure der Roche Diagnostics 
%		GmbH}\EN{Scholarship for Engineers of the Roche Diagnostics GmbH}}
%		{2013}{}{}	
%\cventry{}
%		{\vspace{-0.5cm}}
%		{\DE{Deutschlandstipendium}\EN{Federal Scholarship of Germany}}
%		{10/2012 -- 10/2013}{}{}	
\vspace{\sepspace}
\cventry{}{\vspace{-0.5cm}}
		{University of Hawaii College of Engineering Dean’s List}
		{08/2009 -- 05/2011}{}{
		\DE{Auszeichnung für einen herausragenden Notendurchschnitt (top 
		20\% des Jahrgangs)}
		\EN{Awarded for excellent grades (best 20\% of cohort)}}
	
	
	
\section{\DE{Fähigkeiten}\EN{Skills}}
\cvitem{\DE{Sprachen}\EN{Languages}}
	{\DE{Deutsch (Muttersprache), Englisch (Verhandlungssicher), 
		Französisch (Gute Kentnisse)}\EN{German (native language), English 
		(fluent), 
		French (conversational)}}
\cvitem{IT}{%\DE{Anfänger}\EN{Beginner}: 
	Matlab, Python (Tensorflow, PyTorch), C, \CC, git, CUDA, Jenkins, Docker, 
	ROS, 
	Latex
	%\DE{Fortgeschritten}\EN{Intermediate}:
	 %\DE{Experte}\EN{Expert}: 
	} 
%\cvitem{\DE{Lizenzen}\EN{Licences}}{\DE{Führerschein B, 
%Sportpilot Gleitschirm, Übungsleiter (Volleyball) und 
%Vereinsmanager C}\EN{Driver's license B, sport pilot 
%(paragliding), volleyball trainer and club manager 
%C-level}}

%\section{References}
%\begin{cvcolumns}
%  \cvcolumn{Category 1}{\begin{itemize}\item Person 1\item Person 2\item 
%Person 
%  3\end{itemize}}
%  \cvcolumn{Category 2}{Amongst others:\begin{itemize}\item Person 1, and\item 
%  Person 2\end{itemize}(more upon request)}
%  \cvcolumn[0.5]{All the rest \& some more}{\textit{That} person, and 
%  \textbf{those} also (all available upon request).}
%\end{cvcolumns}

% Publications from a BibTeX file without multibib
%  for numerical labels: 
%\renewcommand{\bibliographyitemlabel}{\@biblabel{\arabic{enumiv}}}% CONSIDER 
%%MERGING WITH PREAMBLE PART
%  to redefine the heading string ("Publications"): 
\renewcommand{\refname}{\EN{Publications}\DE{Veröffentlichungen}}

\nocite{*}
%\bibliographystyle{plain}
%\bibliography[type=article]{mybib}                        % 'publications' is 
%the name 
%of a BibTeX file

\section{\EN{Publications}\DE{Veröffentlichungen}}
\printbibliography[heading={subbibliography},title={\DE{Artikel in 
		Zeitschriften}\EN{Journal Articles}},type=article]
\printbibliography[heading={subbibliography},title={\DE{Konferenzbeiträge}\EN{Conference
		Proceedings}},type=inproceedings]

%\printbibliography
% Publications from a BibTeX file using the multibib package
%\section{Publications}
%\nocitebook{book1,book2}
%\bibliographystylebook{plain}
%\bibliographybook{publications}                   % 'publications' is the name 
%%of a BibTeX file
%\nocitemisc{misc1,misc2,misc3}
%\bibliographystylemisc{plain}
%\bibliographymisc{publications}                   % 'publications' is the name 
%%of a BibTeX file

\clearpage

\end{document}
